\section{Deployment View}
%%Contents
% This view describes the environment within which the system is executed. It describes the geographic 
% distribution of the system or the structure of the hardware components that execute the software. It 
% documents workstations, processors, network topologies and channels, as well as other elements of the 
% physical system environment. The deployment view shows the system from the operator’s point of view.
% Please explain how the systems’ building blocks are aggregated or packaged into deployment artifacts 
% or deployment units.

%%Motivation
% Software is not much use without hardware. The minimum that is needed by you as a software architect 
% is sufficient detail of the underlying (hardware) deployment so that you can assign each software 
% building block that is relevant for the system’s operations to some hardware element. (This also 
% holds for any COTS that is a prerequisite for the operations of the overall system.) These models 
% should enable the operator to properly install the software.

%%Form
% The UML provides deployment diagrams for describing this view. Use these – possibly in a nested 
% manner if necessary. (The top level deployment diagram should already be part of your context view, 
% showing your infrastructure as a single black box (cf. chapter 3.2). Here you are zooming into this 
% black box with additional deployment diagrams.)
% Diagrams by your hardware-oriented colleagues who describe processors and channels are also usable. 
% You should abstract these to aspects relevant for software deployment.

\subsection{Infrastructure Level  1}
% - Shows the deployment of the overall system to 1 – n processors or sites as well as the physical connections among these elements.
% - Lists the most important reasons that led to this deployment structure, i.e. the specific selection of nodes and channels.
% - Should also mention rejected alternatives incl. reasons for their rejection.

\subsubsection{Processor  1}
% < insert node template here>

% node template:
% - Description
% - Deployed software building blocks
% - Quality attributes (performance, constraints, ...)
% - Other administrative information
% - Open issues

\subsubsection{Processor n}
% < insert node template here>

\subsubsection{Channel  1}
%% Contents
% Specification of the channel’s attributes, as relevant for software architecture.

%% Motivation
% Specify at least those attributes of the communications channels that you need for proving 
% fulfillment of non-functional requirements such as maximal throughput, probability for faults, etc.

%% Form
% Often you will refer to a standard (e.g. CAN-Bus, 10Mbit Ethernet, IEEE 1394, ...).
% If you need more information use a structure similar to the node template (especially to document
% quality aspects of channels like throughput, error rates, ...) 

\subsubsection{Channel  n}

\subsection{Infrastructure Level 2}
% Contents
% Additional deployment diagrams with similar structure as above, refining each node of infrastructure 
% level 1 that needs more details to map the software blocks.

%% Motivation
% To describe additional details of the infrastructure, as needed by software deployment.

