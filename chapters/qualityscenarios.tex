\section{Quality Scenarios}
% This chapter summarizes all you (or other stakeholders) might need to systematically 
% evaluate the architecture against the quality requirements.
% It contains:
% - Quality Tree (sometimes called utility tree), an overview of the quality requirements 
% - Evaluation or quality scenarios - detailed descriptions of the quality requirements or goals.

\subsection{Quality Tree}
%% Content
% The quality tree ( as defined in ATAM – Architecture Tradeoff Analysis Method) with quality 
% / evaluation scenarios as leafs.

%% Motivation
% When you want to evaluate the quality (especially risks to certain quality attributes) with 
% methods like ATAM, you need to systematically refine your quality goals (from chapter 1.2). 
% The quality tree shows the top-down refinement of the stakeholder-specific notion of 
% quality.

%% Form
% We personally prefer mind maps to a pure tree-like structure, as mind maps allow arbitrary 
% cross-references between scenarios, attributes and intermediate nodes.
% Often it is difficult to assign scenarios to single quality attributes, as the scenario 
% refers to several qualities at once. Simply draw references from such scenarios to all 
% affected nodes!

\subsection{Evaluation Scenarios}
%% Contents
% Scenarios describe a system’s reaction to a stimulus in a certain situation. They thus 
% characterize the interaction between stakeholders and the system. Scenarios operationalize 
% quality criteria and turn them into measurable quantities.
% Two scenarios are relevant for most software architects:
% - Usage scenarios (also called application scenarios or use case scenarios) the system’s runtime reaction to a certain stimulus. This also includes scenarios that describe the system’s efficiency or performance. Example: The system reacts to a user’s request within one second.
% - Change scenarios describe a modification of the system or of its immediate environment. Example: Additional functionality is implemented or requirements for a quality attribute change.

% If you design safety critical systems a third type of scenarios is important for you:
% - Boundary or stress scenarios describe how the system reacts to exceptional conditions. Examples: How does the system react to a complete power outage, a serious hardware failure, etc.

Figure: Schematic depiction of scenarios (cf. [Bass+03])
% Scenarios comprise the following major parts (according to [Starke05], original structure from [Bass+03]):
% - Stimulus: Describes a specific interaction between the (stimulating) stakeholder and the system. Example: A user calls a functions, a developer implements an extension, an administrator installs or configures the system.
% - Source of the stimulus: Describes where the stimulus comes from. Examples: internal or external, user, operator, attacker, manager.
% - Environment: Describes the system’s state at the time of arrival of the stimulus. This should list all preconditions that are necessary for comprehension of the scenario. Examples: Is the system under normal or maximal load? Is the data base available or down? Are any users online?
% - System artifact: Describes the part of the system is affected by the stimulus. Examples: The whole system, the data base, the web server.
% - System response: Describes the system’s reaction to the stimulus as determined by the architecture. Examples: Is the function called by the user executed. How long does the developer need for implementation? Which parts of the system are affected by the installation / configuration?
% - Response measure: Describes how the response can be measured or evaluated. Examples: Downtime in hours, correctness yes/no, time for code change in person days, reaction time in seconds.

%% Motivation
% You need scenarios for the evaluation and review of architectures. They take the role of a
% “benchmark” and aid in measuring the architecture’s achievement of its objectives
% regarding the non-functional requirements and quality attributes.

%% Form
% Tabular or free text. Explicitly highlight the scenario’s elements (source, environment, 
% artifact, response, measure). 
% Alternatively, use similar notations as those suggested in 6. Runtime View.

%% Background Information
% There are relations between scenarios and the runtime view. Often you can use scenarios of 
% the runtime view fully or as a basis for evaluation. Evaluation scenarios additionally 
% contain response measures that are often not considered in the pure execution focus of 
% runtime scenarios.
