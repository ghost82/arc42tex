\section{Introduction and Goals}
% The introduction to the architecture documentation should list the driving forces that software 
% architects must consider in their decisions. This includes on the one hand the fulfillment of functional
% requirements of the stakeholders, on the other hand the fulfillment of or compliance with required %constraints, always in consideration of the architecture goals.


\subsection{Requirements Overview}
% Short description of the functional requirements.
% Digest (or abstract) of the requirements documents.
% Reference to complete requirements documents (incl. version identification and location).

%% Contents
% A compact summary of the functional environment of the system. Answers the following questions (at least approximately):
% - What happens in the system’s environment?
% - Why should the system exist? Why is the system valuable or important? Which problems does the system solve?

%% Motivation
% From the point of view of the end users a system is created or modified to improve execution of a business activity.
% This essential architecture driver must not be neglected even though the quality of an architecture is mostly judged by its level of fulfillment of non-functional requirements.

%% Form
% Short textual description, probably in tabular use-case format.
% The business context should in any case refer to the corresponding requirements documents.

%% Examples
% Short descriptions of the most important:
% - business processes
% - functional requirements
% - non-functional requirements and other constraints (the most important ones must be covered as architecture goals or are listed in the “Constraints” section), and/or
% - quantity structures
% - background information
% Here you can reuse parts of the requirements documents – but keep these excerpts short and 
% balance %readability against avoidance of redundancy.


\subsection{Quality Goals}
%% Contents
% The top three (max five) goals for the architecture and/or constraints whose fulfillment is 
% of highest %importance to the major stakeholders.
% Goals that define the architecture’s quality could be:
% - availability
% - modifiability
% - performance
% - security
% - testability
% - usability

%% Motivation
% If you as an architect do not know how the quality of your work can be judged …

%% Form
% Simple tabular representation, ordered by priorities

% Background Information
% NEVER start developing an architecture if these goals have not been put into writing and
% have not been %signed by the major stakeholders.

% We have endured projects lacking defined quality goals much too often. We do not like to 
% suffer, therefore %we are by now highly convinced that the few hours spent on collecting
% quality goals are well invested.
% PH & GS.

%% Sources
% The DIN/ISO 92000 Standard contains an extensive set of possible quality goals.
% Or use chapters 10 – 17 of the VOLERE template as a starting point (www.volere.co.uk).
% PH



\subsection{Stakeholders}
%% Contents
% A list of the most important persons or organizations that are affected by can contribute 
% to the architecture.

%% Motivation
% If you do not know the persons participating in or concerned with the project you may get 
% nasty surprises later in the development process. Should your project manager maintain 
% this list, make sure that all the people influencing the architecture are part of it.

%% Form
% Simple table with role names, person names, their knowledge as pertaining to architecture, 
% their availability, etc.
