\section{Building Block View}
%% Contents
% Static decomposition of the system into building blocks (modules, components, 
% subsystems, subsidiary systems, classes, interfaces, packages, libraries, 
% frameworks, layers, partitions, tiers, functions, macros, operations, data 
% structures, …) and the relationships thereof.

%% Motivation
% This is the most important view, that must be part of each architecture 
% documentation. In building construction this would be the floor plan.

%% Form
% The building block view is a hierarchical collection of black box and white box 
% descriptions as shown in the following diagram:

% Level 1 contains the white box description of the overall system (system under 
% development / SUD) made up of black box descriptions of the system’s building blocks.
% Level 2 zooms into the building blocks of Level 1 and is thus made up of the white 
% box descriptions of all building blocks of Level 1 together with the black box 
% descriptions of the building blocks of Level 2.
% Level 3 zooms into the building blocks of Level 2, etc.
% The section is structured as follows:
% ============================

%% White Box Template:
% Contains multiple building blocks with corresponding black box descriptions.
% One or more black box templates:
% Each building block appearing in the white box template should be described as 
% follows:
% - Purpose / Responsibility:
% - Interface(s):
% - Implemented requirements:
% - Variability: 
% - Performance attributes:
% - Repository / Files:
% - Other administrative information: Author, Version, Date, Revision History 
% - Open issues:


\subsection{Level 1}
% Here you describe the white box view of level 1 according to the white box 
% template. The structure is given below.
% The overview diagram describes the inner structure of the overall system in terms  
% of building blocks 1 – n, as well as their relationships and interdependencies.
% It is also useful to list the most important reasons that led to this structure, 
% esp. as relevant to the interdependencies / relationships among the building blocks 
% at this level.
% You should also mention rejected alternatives incl. reasons for their rejection.
% The following diagram shows the main building blocks of the system and their
% interdependencies:
% <insert overview diagram here>
%
% Comments regarding structure and interdependencies at Level 1:


\subsubsection{Building Block Name 1 (Black Box Description)}
% Structure according to black box template:
% - Purpose / Responsibility:
% - Interface(s):
% - Implemented requirements:
% - Variability: 
% - Performance attributes:
% - Repository / Files:
% - Other administrative information: Author, Version, Date, Revision History 
% - Open issues:

% <insert the building block’s black box template here>


\subsubsection{ Building Block Name n (Black Box Description)}

% <insert the building block’s black box template here>

\subsection{Level 2}
% Describe all building blocks comprising level 1 as a series of white box templates. 
% The structure is given below for three building blocks and should be duplicated as 
% needed.

\subsubsection{Building Block Name 1 (White Box Description)}
% Shows the inner workings of the building block in form of a diagrams with local 
% building blocks 1 – n, as well as their relationships and interdependencies.
% It is also useful to list the most important reasons that led to this structure, 
% esp. as relevant to the interdependencies / relationships among the building blocks 
% at this level.
% You should also mention rejected alternatives incl. reasons for their rejection.

% <insert diagram of building block 1 here>

\subsubsection{Building Block Name 1.1 (Blac Box Description)}
% Structure according to black box template:
% - Purpose / Responsibility:
% - Interface(s):
% - Implemented requirements:
% - Variability: 
% - Performance attributes:
% - Repository / Files:
% - Other administrative information: Author, Version, Date, Revision History 
% - Open issues:

\subsubsection{Building Block Name 1.n (Black Box Description)}
% Structure according to black box template

\subsubsection{Description of Relationships}

\subsubsection{Open Issues}


\subsection{Level 3}
% Describe all building blocks comprising level 2 as a series of white box templates. 
% The structure is identical to the structure of level 2. Duplicate the corresponding 
% sub-sections as needed.
% Simply use this section structure for any additional levels you would like to
% describe.

