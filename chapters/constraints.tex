\section{Architecture Constraints}
%%Contents
%Any requirement that constrains software architects in their freedom of design decisions or the development process.

%%Motivation
%Architects should know exactly where they are free in their design decisions and where they must adhere to constraints.
%Constraints must always be dealt with; they may be negotiable, though.

%%Form
%Informal lists, structured by the sub-sections of this section.

%%Examples
%see sub-sections
%Background Information
%In the optimal case constraints are defined by requirements. In any case, at least the architects must be aware of constraints.
%The influence of constraints is described by [Hofmeister et al] (Software Architecture, A Practical Guide, Addison Wesley 1999)  using the term “global analysis”.

\subsection{Technical Constraints}
%%Contents
%List all technical constraints in this section. This category covers hard- and software infrastructure, %applied technologies (operating systems, middleware, databases, programming languages, …).


\subsubsection{Hardware-Constraints}

%<insert constraint here> 

\subsubsection{Software-Constraints}

%<insert constraint here>

\subsubsection{Operating System Constraints}

%<insert constraint here>

\subsubsection{Programming Constraints}

%<insert constraint here>

%
%%Examples
%
\begin{center}
    \begin{tabular}{| l | p{6cm}  |}
		\hline
		\textbf {Constraint} & \textbf {Description} \\ \hline
		Hardware infrastructure & Processors, memory, networks, firewalls and other relevant elements\\ \hline
		Software infrastructure & Operating systems, database systems, middleware, communications systems, transaction monitors, web servers, directory services\\ \hline
		System operations & Batch- or online operations of the system or of required external systems?\\ \hline
		Availability of the runtime environment & Data center with 7x24 uptime? Will there be service times that cause reduced availability of the system or important parts thereof?\\ \hline
		Graphical user interface  & Are there any restrictions related to GUI (style guide)?\\ \hline
		Libraries, frameworks, components & Is there any COTS that must be used?\\ \hline
		Programming languages  & Object oriented, structured, declarative, or rule-based languages? Compiled or interpreted languages?\\ \hline
		Reference architectures  &  Are there any comparable or reusable reference projects in the organization?\\ \hline
		Analysis and design methodologies  & Object oriented or structured methodologies?\\ \hline
		Data structures & Requirements for certain data structures, interfaces to existing databases or files?\\ \hline
		Software interfaces & Interfaces to existing applications\\ \hline
		Programming requirements  & Programming guidelines, fixed program structure\\ \hline
		Technical communications  & synchronous or asynchronous; protocols\\ \hline
    \end{tabular}
\end{center}

\subsection{Organizational Constraints}
%%Contents
% Enter all organizational, structural, and resource-related constraints. This
% category also covers standards and legal constraints that you must comply with.

\subsubsection{Organization and Structure}

%<insert constraint here> 

\subsubsection{Resources (Budget, Time, Personnel)}

%<insert constraint here>

\subsubsection{Organizational Standards}

%<insert constraint here>
\subsubsection{Legal Factors}

%<insert constraint here>

%
%%Examples
%
\begin{center}
    \begin{tabular}{| p{6cm} | p{6cm}  |}
		\hline
		\textbf {Constraint} & \textbf {Description} \\ \hline
		Organization and Structure \\ \hline
		Sponsor’s organizational structure & Potential changes of responsibilities? Changes of contact persons? \\ \hline
		Project team’s organizational structure & with/without subcontractors decision-making power of the project manager \\ \hline
		Decision makers & Experience with similar projects \\ \hline
		Existing partnerships or co-operations & Are there any co-operations between the organizations and certain software companies? Such partnerships often influence procurement decisions (independent of system requirements).\\ \hline
		Internal development or outsourcing & Develop internally or outsource to external service companies? \\ \hline
		Development of a product or for internal use? & Implies different processes in requirements analysis and decision making. In the case of product development: New product for a new market? Improved product for an existing market? Productizing of an existing (internal) system? Development for internal use only? \\ \hline
    \end{tabular}
\end{center}
\begin{center}
    \begin{tabular}{| p{6cm} | p{6cm}  |}
		\hline
		\textbf {Constraint} & \textbf {Description} \\ \hline
		Resources (Budget, Time, Personnel) \\ \hline
		Fixed price or time/effort? & Is the project’s budget fixed or is it calculated by time or effort? \\ \hline
		Schedule & Is the schedule flexible? Is there a fixed delivery date? Which stakeholders control the delivery date? \\ \hline
		Schedule vs. functionality & What has higher priority: The delivery date or the functionality? \\ \hline
		Release-schedule & At which dates should which functionality be available in which releases / versions? \\ \hline
		Project’s budget & Fixed or flexible? What amount is available? \\ \hline
		Budget for technical resources & Buy or rent development tools? (hardware and software) \\ \hline
		Team & Number of team members, qualifications, motivation, availability. \\ \hline
    \end{tabular}
\end{center}
\begin{center}
    \begin{tabular}{| p{6cm} | p{6cm}  |}
		\hline
		\textbf {Constraint} & \textbf {Description} \\ \hline
		Organizational Standards \\ \hline
		Development process & Requirements concerning development process? This includes internal standards for modeling, documentation and implementation. \\ \hline
		Quality standards & Is the organization required to adhere to quality standards (such as ISO-9000)? \\ \hline
		Development tools & Requirements related to development tools (such as CASE-Tool, database, IDE, communications software, middleware, transaction monitor). \\ \hline
		Configuration and version management & Requirements concerning processes and tools \\ \hline
		Test tools and processes & Requirements concerning processes and tools \\ \hline
		Acceptance- and release processes & Data modeling and database design User interfaces Business processes (workflow) Usage of external systems (e.g. write access to external databases) \\ \hline
		Service Level Agreements & Requirements or standards related to availability or required service levels? s\\ \hline
    \end{tabular}
\end{center}
\begin{center}
    \begin{tabular}{| p{6cm} | p{6cm}  |}
		\hline
		\textbf {Constraint} & \textbf {Description} \\ \hline
		Legal Factors \\ \hline
		Liability & Are there any legal aspects related to usage or operations of the system? Could the system cause loss of human life or hazard to human health? Could the system impact the operations of external systems or businesses? \\ \hline
		Data privacy and security & Does the system store or process any data worthy of protection \\ \hline
		Auditing & Are any aspects of the system under legal obligation to present evidence? \\ \hline
		Aspects of international law & Will the system be used in an international context? Are there varying constraints on system usage in different countries (example: use of encryption technology)? \\ \hline
    \end{tabular}
\end{center}

\subsection{Conventions}
%%Contents
%List all conventions that are relevant for the development of your software architecture.

%%Form
%Either insert the conventions directly in this document or refer to other documents.

%%Examples
% - Coding guidelines
% - Documentation guidelines
% - Guidelines for version and configuration management
% - Naming conventions
