\section{System Scope and Context}
%% Contents
% The context view defines the boundaries of the system under development to
% distinguish it from neighboring systems. It thereby identifies the system’s relevant
% external interfaces.
% Make sure that the interfaces are specified with all their relevant aspects (what is 
% communicated, in which format is it communicated, what is the transport medium, …), 
% even though some popular diagrams (such as the UML use case diagram) represent only 
% a few aspects of the interface.

%% Motivation
% The interfaces to neighboring systems are among most critical and risky aspects of a 
% project. Ensure early on that you have understood them in their entirety.

%% Form
% - Various context diagram (see below)
% - Lists of neighboring systems and their interfaces.


\subsection{Business Context}
%% Contents
% Identify all neighboring systems and specify all logical/business data that is 
% exchanged with the system under development. Add data formats and communication 
% protocols with neighboring systems and the general environment if these are not 
% specified in detail with the relevant components.

%% Motivation
% Understanding the information exchange with neighboring systems (i.e. all input 
% flows and all output flows).

%% Form
% Logical context diagram.
% In UML this can be simulated e.g. by class diagrams, use case diagrams, 
% communications diagrams – i.e. all diagrams that represent the system as a black box 
% and explain its interfaces to neighboring systems (in varying degrees of detail).
% If there are many neighboring systems you can substitute the context diagram with a 
% table, including all the neighboring systems, their inputs and their outputs.

% We often tend to a pragmatic approach – but here we insist on a list of all (a-l-l) 
% neighboring systems. Too many projects have failed because they were not aware of 
% their neighbors. 


\subsection{Technical Context}
%% Contents
% Specification of the communication channels linking your system to neighboring 
% systems and the environment.

%% Motivation
% Understanding of the media used for information exchange with neighboring systems, 
% and the environment.

%% Form
% E.g. UML deployment diagram describing channels to neighboring systems, together 
% with a mapping table of logical input and output flows of the logical context 
% diagram (3.1) to the channels.

\subsection{External Interfaces}

% For many building blocks you can describe its interfaces directly in the black box 
% template of the building block. For complex interfaces – and external interfaces are 
% normally very complex – it is worth while to describe them in separate sections. Use 
% the following interface template to guide you towards many questions that might be 
% relevant for the interface. 

\subsubsection{Interface Id}
%
\begin{center}
    \begin{tabular}{| l | p{6cm}  |}
		\hline
		\textbf {Name} & \textbf {<name of Interface>} \\ \hline
		Version&\\ \hline
		Changes w.r.t previous release&\\ \hline
		Who changed it and why?&\\ \hline
		Contact person&\\ \hline
	\end{tabular}
\end{center}
%

\subsubsection{Business Context of the Interface}
%

\subsubsection{Business Processes}
% <Diagram or desciption of business processes relevant for this interface>

\subsubsection{Interface Data}
% <Description of interface data>
% Technical Context
% Form of interaction

\subsubsection{Requirements for the Interface}

\subsubsection{Security Aspects}

\subsubsection{Quantities}
% Runtime
% Throughput/Volume
% Availability
% Logging
% Archiving

\subsubsection{Participating Resources}
% 

\subsubsection{Syntax: Data and Formats}
% Data Formats
% Validity \& Plausibility Rules
% Encoding, Character Sets
% Configuration data

\subsubsection{Syntax: Methods/Functions}
% Check data

\subsubsection{Interface Process}
% Logical and technical processes

\subsubsection{Semantics}
% Side effects, consequences

\subsubsection{Technical Infrastructure}
% Technical protocols

\subsubsection{Error and Exception Handling}
% 

\subsubsection{Constraints and Assumptions}
% Access Rights
% Temporal constraints
% Parallel Access
% Preconditions for using the interface

\subsubsection{Operating the Interface}
% 

\subsubsection{Meta Information for the Interface}
% Person in charge
% Costs of using the interface
% Organizational Issues
% Versioning

\subsubsection{Examples of Using the Interface}
% Sample data
% Sample flows and interactions
% Programming Examples

\subsubsection{External Interface 1}
% <insert interface template>

\subsubsection{External Interface N}
%  <insert interface template>
