\section{Runtime View}
%% Contents
% alternative terms:
% - Dynamic view
% - Process view
% - Workflow view
% This view describes the behavior and interaction of the system’s building blocks as runtime elements 
% (processes, tasks, activities, threads, …).
% Select interesting runtime scenarios such as:
% - How are the most important use cases executed by the architectural building blocks?
% -  Which instances of architectural building blocks are created at runtime and how are they started,  controlled, and stopped.
% - How do the system’s components co-operate with external and pre-existing components?
% - How is the system started (covering e.g. required start scripts, dependencies on external systems, databases, communications systems, etc.)?

% Note: The main criterion for the choice of possible scenarios (sequences, workflows) is their 
% architectural relevancy. It is not important to describe a large number of scenarios. You should 
% rather document a representative selection.
% Candidates are:
% 1.	The top 3 – 5 use cases
% 2.	System startup
% 3.	The system’s behavior on its most important external interfaces
% 4.	The system’s behavior in the most important error situations

%% Motivation
% Esp. for object-oriented architectures it is not sufficient to specify the building blocks with their 
% interfaces, but also how instances of building blocks interact during runtime.

%%Form
% Document the chosen scenarios using UML sequence, activity or communications diagrams. Enumerated 
% lists are sometimes feasible.
% Using object diagrams you can depict snapshots of existing runtime objects as well as instantiated 
% relationships. The UML allows to distinguish between active and passive objects.
\subsection{Runtime Scenario 1}
% - Runtime diagram (or other adequate description of scenario!)
% - Description of the notable aspects of the interactions between the building block instances depicted in this diagram.

\subsection{Runtime Scenario 2}
% - Runtime diagram  (or other adequate description of scenario!)
% - Description of the notable aspects of the interactions between the building block instances depicted in this diagram.

\subsection{Runtime Scenario n}
% - Runtime diagram  (or other adequate description of scenario!)
% - Description of the notable aspects of the interactions between the building block instances depicted in this diagram.
